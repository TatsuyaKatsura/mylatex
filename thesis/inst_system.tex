
%既存のシステムはユーザが実際に操作するフロントエンドと,フロントエンドから受け取ったデータを処理するバックエンドで構成される.その詳細を以下の\ref{sec:2-1-1}節,\ref{sec:2-1-2}節で説明する.
Existing system is a front that the user actually operates. The system consists of a front end and a back end that processes data received from the front end.
The details are described in sections 2.1.1 and 2.1.2 below.

%\subsubsection{フロントエンド}
\subsubsection{Front-end}
%フロントエンドは主にJavaScriptのフレームワークReactを用いて構築しており,以下のような機能を有している.
The frontend is mainly built using the JavaScript framework React, and has the following functions.

\begin{description}
   %\item[コーパスアップロード]\mbox{}\\
   \item[Corpus Upload]\mbox{}\\
%解析対象のコーパスファイルのアップロードを行う.文の区切りは句点,感嘆符お
%よび疑問符で認識される.
Upload the corpus file to be analyzed. Sentence breaks are recognized by punctuation marks, exclamation points
and question marks.
   \item[Display the  results of corpus analysis]\mbox{}\\
%以下の図のようなコーパスの解析結果を表示する.
The following figure shows the analysis results of a corpus such as Figure.



%\item[言語パターン構築]\mbox{}\\
\item[Linguistic pattern construction]\mbox{}\\
%Blocklyを用いて視覚的にブロックを組み合わせることで検索したい言語パターンを構築することがで
%きる.多種のブロックを組み合わせることにより,ASAの解析結果の情報を複雑に
%検索することができる.図\ref{blockly:1}は著者と作品を抽出する言語パターンである.
Using Blockly, you can construct the language pattern you want to search for by visually combining blocks.
By combining many different types of blocks, the information in the ASA analysis results can be complex. By combining many different types of blocks, it is possible to search for information on ASA parsing results in complex ways.
Figure is a language pattern for extracting authors and works.



%また構築した言語パターンを xml 形式のファイルにエクスポートすることができる.エク
%スポートしたファイルをインポートすることもできる.
It is also possible to export the constructed language patterns to xml format files.
The exported file can also be imported.
%\item[言語パターンマッチ結果表示]\mbox{}\\
\item[Display the results of linguistic pattern matching results]\mbox{}\\

%言語パターンマッチの実行結果を表示する.表示形式には,KWIC 形式,強調形式,
%テーブル形式がある.図\ref{blockly:1}の言語パターンを検索クエリとして言語パターンマッチを実行した結果が図\ref{display_result}である.AuthSlockは作品,WorkSlocは著者を意味しており,それぞれの形式でコンコーダンス表示されている.
Displays the results of language pattern matching. The display format includes KWIC format, highlighting format, and
table format. Figure  shows the result of executing language pattern matching using the language pattern in Figure 

\end{description}





%\subsubsection{バックエンド}
\subsubsection{Back-end}
%バックエンドはPython Web フレームワークである Djangoで構築されている. ASAによる解析にはasapy, Prolog処理系にはprologpyを用いて実装されている.
The backend is built on Django, a Python web framework. It is implemented using asapy for analysis by ASA and prologpy for Prolog processing.

\begin{description}
%\item[コーパス解析]\mbox{}\\
\item[Corpus analysis]\mbox{}\\

%解析対象のコーパスを読み込み,ASA による解析, Prolog への変換を行う.
%生成した解析結果をフロントエンドに返却し,ブラウザのメモリに保存する.
Reads the corpus to be parsed, parses it with ASA, and converts it to Prolog.
The generated analysis results are returned to the front end and stored in the browser's memory.
%\item[言語パターンマッチ]\mbox{}\\
\item[Linguistic pattern matching]\mbox{}\\
%対象の Prolog と検索クエリに対し,Prolog 処理系によるクエリ検索を行う.
%生成したパターンマッチ結果をフロントエンドに返却し,ブラウザのメモリに保存する.
Query the target Prolog and search query using the Prolog processor.
The generated pattern matching results are returned to the front end and stored in the browser memory.
\end{description}


